\section{Rapport de travail}\label{rapport-de-travail}

par Rémi Langevin et Christophe Apollon-Roy

\subsection{Problèmes rencontrés}\label{probluxe8mes-rencontruxe9s}

\paragraph{Trouver le format de base de
Llambda}\label{trouver-le-format-de-base-de-llambda}

Au départ, nous avions décidé de transformer l'arbre de Sexp en liste.
Au début, cela facilitait la tâche. Toutefois, nous avons rapidement eu
des difficultés avec les cas plus complexes, dont le lambda. À partir de
ce moment, nous avons décidé d'y aller avec du pattern matching et
déterminer les patterns associés à chaque type de Sexp.

\paragraph{Gérer les Llet imbriqués et avec assignation
multiples}\label{guxe9rer-les-llet-imbriquuxe9s-et-avec-assignation-multiples}

Trouver les patterns associés et comment extraire de multiples
déclarations dans un Llet fut une tâche complexe. Nous avons eu besoin
de dessiner quelques arbres afin de bien généraliser.

\paragraph{Currying}\label{currying}

Le currying nous a couté beaucoup de temps et de modifications dans le
code. À un certain point, d'un côté, nous pouvions faire fonctionner les
fonctions non-curried, de l'autre less fonctions curried, mais jamais
les deux en même temps. Cela se jouait au niveau du Lapp. Sans currying,
nous avions:

\begin{verbatim}
s2l (Scons Snil a) =
   case (s2l a) of
   ...
   (Lapp x y) -> Lapp x y
\end{verbatim}

et avec currying:

\begin{verbatim}
s2l (Scons Snil a) =
   case (s2l a) of
   ...
   (Lapp x y) -> Lapp (Lapp x y) []
\end{verbatim}

Nous avions donc eu du trouble à trouver une façon de faire qui
fonctionne pour les deux.

\paragraph{Unsweetner}\label{unsweetner}

Se débarasser du sucre syntaxique fut une autre fonction complexe à
implémenter sans briser le code déjà existant.

\subsection{Surprises}\label{surprises}

Nous avons été surpris à quel point les cases, cons et if furent facile
à implémenter. Autrement, pouvoir utiliser la structure de l'arbre
directement avec du pattern matching nous a surpris.

De plus, malgré la difficulté que le travail nous a donné et bien que
nous ne le ferions pas à nouveau, nous nous sommes surpris à apprécier
l'exercice.

Ensuite, nous avons été surpris à quel point nous pouvons construire par
dessus les fonctions existantes sans se soucier de comment elle allait
agir lorsque celles-ci sont bien pensées à la base.

Malgré tout cela, les problèmes rencontrés nous ont couté beaucoup de
temps, ce qui a un peu terni l'expérience.

\subsection{Choix}\label{choix}

Nous avons choisi de ne pas avoir de souplesse dans la syntaxe acceptée,
car cela devenait trop complexe de trouver les patterns associés à
certains modèles. Nous avons choisi, en fin de travail, de ne faire
aucune vrai différence entre les dlet et les llet dans eval (s2l fait la
différence), donc au final, il n'y a aucune différence entre les deux.

\subsection{Options sciemment
rejetées}\label{options-sciemment-rejetuxe9es}

Nous avons sciemment rejeté l'idée de v
